%!TEX root = ../memoria.tex

\chapter{Requisitos no funcionales}

Se hace menester comenzar el plan de calidad definiendo los “atributos de calidad”, los indicadores de medición y los valores de validación de estos. Lo anterior, con el fin de definir los márgenes generales en el desarrollo de la aplicación y el nivel de integración del sistema con los atributos antes mencionados.

\section{Definición de atributos de calidad}

A continuación, se procederá con la presentación y definición de los requisitos no funcionales que definirán el lineamiento del desarrollo de la aplicación.

\begin{description}

\item[Fiabilidad:]\hfill

Se debe garantizar la respuesta de la aplicación ante una acción del usuario y además que la información entregada sea correcta.

\item[Usabilidad:]\hfill

Facilidad en el uso de la aplicación, reflejándose en la rapidez para aprender el funcionamiento del sistema y recordar este aprendizaje.

\item[Adapabilidad:]\hfill

Capacidad de la aplicación de funcionar en distintos entornos informáticos.

\item[Temporización:]\hfill 

Debe emplear el menor tiempo posible en entregar una respuesta visible al usuario ante una acción de este.

\item[Soportar alta carga:]\hfill 

El Sistema debe ser capaz de funcionar incluso en momentos de alto tráfico. 

\item[Disponibilidad:]\hfill 

El sistema debe encontrarse disponible para su acceso o consulta el mayor tiempo posible.

\item[Seguridad:]\hfill 

Protección de la infraestructura e información contenida en el sistema.

\item[Escalabilidad:]\hfill 

Capacidad del sistema de aumentar la carga de trabajo o sus funciones sin ver afectado el funcionamiento del resto de funcionalidades y tampoco disminuir su calidad.

\item[Mantenibilidad:]\hfill

 Esfuerzo y tiempo dedicado al soporte del sistema.

\end{description}

\section{Objetivos Cuantificables}

En la presente sección se establecerán los indicadores de cada uno de los atributos de calidad definidos en el apartado anterior. Además, se establecen los valores mínimos y máximos de los indicadores de los atributos de calidad, con el propósito de establecer de forma objetiva el cumplimiento requisitos no funcionales deseados en el sistema.

\subsection{Fiabilidad}

Es necesario que la información entregada por cualquiera de los distintos módulos que forman \emph{Gerprin} sea legible por los módulos restantes y de este modo no tener perdidas en la comunicación entre distintas plataformas. Las solicitudes y respuesta deben ser verídicas y corresponder a acciones reales desencadenadas por los actores de la aplicación.

Para lo anterior es necesario que el sistema de comunicación que se desarrollará no presente perdidas de información, sin embargo, se deben fijar estándares distintos para cada módulo respecto a la fiabilidad de los datos, debido a que la información que maneja la aplicación tiene distintos grados de importancia, incluso permitiendo perdida parcial de esta en determinadas condiciones.

Comenzando con el sistema a bordo, es necesaria la comunicación fiable con la \emph{API} de \emph{Gerprin} al momento de realizar activación o desactivación del cortacorriente mediante la aplicación móvil, pero no lo es para establecer una conexión fiable cada vez que el módulo contenedor del \emph{GPS} envía la posición. Por este motivo el envío de la ubicación se realiza mediante protocolo \emph{UDP}\footnote{https://www.ietf.org/rfc/rfc768.txt} , a diferencia del resto de las funciones del sistema a bordo que deben emplear \emph{TCP}\footnote{https://tools.ietf.org/html/rfc793}.

Por otro lado, la aplicación móvil y la \emph{API} desarrollada, deben contar con una conexión fiable de comunicación, esto debido a la información critica que es manejada por estos módulos.

\subsubsection{Objetivos cuantificables}

\begin{itemize}
	\item
	Funcionalidades de la aplicación móvil sin errores.
	\item
	Servicios entregados por la \emph{API} sin errores.
	\item
	Envió de posición de forma precisa e intervalos regulares sin errores.
	\item
	Funcionalidad de activación y desactivación de relé de forma remota sin errores.
\end{itemize}

\subsubsection{Criterios de medición.}

\begin{table}[H]
    \caption[Validación para los indicadores de fiabilidad.] {Validación para los indicadores de fiabilidad.}
    \label{tbl:Criterios de Validación fiabilidad}
    \begin{tabular}{|p{.6\textwidth}|p{.34\textwidth}|}
        \hline
        \textbf{Criterio} &  \textbf{Medición}\\
    	\hline
    	\hline
    	Funcionalidades de la aplicación móvil sin errores. & 95\% de efectividad sin errores. \\ \hline
		Servicios entregados por la \emph{API} sin errores. & 95\% de respuestas sin errores. \\ \hline
		Envió de posición de forma precisa e intervalos regulares sin errores. & 75\% de posiciones almacenadas con un error de ±20 metros. \\ \hline
		Funcionalidad de activación y desactivación de relé de forma remota sin errores. & 95\% de efectividad. \\
        \hline
    \end{tabular}
\end{table}

Los criterios de validacion se basan en la experiencia de aplicaciones similares. Es necesario considerar una conexión a internet estable que permita el envío y recepción de información.

\subsection{Usabilidad}

Es necesario para el módulo de la aplicación móvil del sistema contar con una interfaz que entregue una curva de aprendizaje empinada, es decir, que permita comprender el funcionamiento de la aplicación móvil en un corto periodo de tiempo.

Para esto es necesario que la interfaz aplique colores y símbolos coherentes con las acciones a realizar o información que muestra, Información ordenada y fácil de entender.

Dado que el sistema será desarrollado para dispositivos móviles es importante que sea responsivo. Por otro lado, el tamaño de la pantalla del dispositivo, si bien, no es importante en términos de la distribución de los elementos a mostrar, si lo es en términos de usabilidad, es decir, que se muestre adecuadamente los elementos en la pantalla y con un tamaño adecuado. Por este motivo se considera que debe adecuarse a la mayor cantidad de dispositivos.

Considerando que el primer \emph{iPhone} contaba con un tamaño de pantalla de 3,5 pulgadas y que estas van en aumento, se considerará esta como el tamaño mínimo de pantalla.

Respecto a la resolución mínima de pantalla, se puede apreciar en el gráfico \ref{chart:Screen Resolution} que aproximadamente el 74\% de los usuarios de dispositivos móviles en chile emplean una resolución mayor a 360 x 640 píxeles.

\begin{figure}[H]
	\centering
	\caption[Screen Resolution, Market Share in Chile Sept 2017 - Sept 2018.]{Screen Resolution, Market Share in Chile Sept 2017 - Sept 2018 \\ http://gs.statcounter.com/screen-resolution-stats/mobile-tablet/chile/\#monthly-201709-201809-bar.}
	\begin{bchart}[step=10, max=100, width=.8\textwidth, unit=\%]
		\bcbar[label=360x640]{55.13}
		\bcbar[label=375x667]{6.68}
		\bcbar[label=320x534]{5.67}
		\bcbar[label=320x568]{4.83}
		\bcbar[label=320x570]{3.74}
		\bcbar[label=640x360]{2.09}
		\bcbar[label=768x1024]{1.79}
		\bcbar[label=414x736]{1.54}
		\bcbar[label=412x732]{1.37}
		\bcbar[label=360x740]{0.88}
		\bcbar[label=720x1280]{0.85}
		\bcbar[label=320x480]{0.76}
		\bcbar[label=360x720]{0.73}
		\bcbar[label=480x800]{0.71}
		\bcbar[label=600x1024]{0.64}
		\bcbar[label=412x846]{0.63}
		\bcbar[label=424x753]{0.6}
		\bcbar[label=480x854]{0.6}
		\bcbar[label=320x569]{0.56}
		\bcbar[label=1024x600]{0.5}
		\bcbar[label=Other]{9.7}
	\end{bchart}
	\label{chart:Screen Resolution}
\end{figure}

\subsubsection{Objetivos cuantificables}

\begin{itemize}
	\item
	Dimensiones de la página adaptables a los tamaños de pantalla de los dispositivos.
	\item
	Dimensiones de la página adaptables distintas resoluciones.
\end{itemize}

\subsubsection{Criterios de Validación}

\begin{table}[H]
    \caption[Validación para los indicadores de usabilidad.] {Validación para los indicadores de usabilidad.}
    \label{tbl:Criterios de Validación usabilidad}
    \begin{tabular}{|p{.6\textwidth}|p{.34\textwidth}|}
        \hline
        \textbf{Criterio} &  \textbf{Medición}\\
    	\hline
    	\hline
    	Tamaño pantalla mínimo. & 3,5 pulgadas. \\ \hline
		Resolución mínima.  & 360x640 pixeles. \\ 
        \hline
    \end{tabular}
\end{table}

\subsection{Adapabilidad}

En vista de la existencia de distintas plataformas móviles, es necesario asegurar un correcto funcionamiento con los sistemas operativos más comunes de estas. A continuación, se puede observar en la gráfica \ref{chart:OS} el detalle en porcentajes del mercado que poseen los sistemas operativos más populares del último año.

\begin{figure}[H]
	\centering
	\caption[OS, Market Share in Chile Sept 2017 - Sept 2018.]{OS, Market Share in Chile Sept 2017 - Sept 2018 \\ http://gs.statcounter.com/os-market-share/mobile-tablet/chile/\#monthly-201709-201809-bar.}
	\label{chart:OS}
	\begin{bchart}[step=10, max=100, width=.8\textwidth, unit=\%]
		\bcbar[label=Android]{81.72}
		\bcbar[label=iOS]{17.55}
		\bcbar[label=Samsung]{0.39}
		\bcbar[label=Windows]{0.2}
		\bcbar[label=Unknown]{0.03}
		\bcbar[label=Playstation]{0.02}
		\bcbar[label=BlackBerry OS]{0.02}
		\bcbar[label=Series 40]{0.01}
		\bcbar[label=Linux]{0.01}
		\bcbar[label=Other]{0.03}
	\end{bchart}
\end{figure}

Como se puede apreciar en el gráfico \ref{chart:OS}, los sistemas más populares corresponden a \emph{Android} e \emph{iOS}, que en conjunto corresponden al 95,26\% del mercado. Teniendo en cuenta estas cifras y el amplio segmento que abarcan estos dos sistemas, el desarrollo de \emph{Gerprin} se concentrará en dichas plataformas. Para el caso de \emph{Android}, el uso de las distintas versiones se muestra en el grafico \ref{chart:Android Version}.

\begin{figure}[H]
	\centering
	\caption[Android Version, Market Share in Chile Sept 2017 - Sept 2018.]{Android Version, Market Share in Chile Sept 2017 - Sept 2018 \\ http://gs.statcounter.com/android-version-market-share/mobile-tablet/chile/\#monthly-201709-201809-bar.}
	\label{chart:Android Version}
	\begin{bchart}[step=10, max=100, width=.7\textwidth, unit=\%]
		\bcbar[label=6.0 Marshmallow]{34.81}
		\bcbar[label=7.0 Nougat]{23.58}
		\bcbar[label=5.1 Lollipop]{17.11}
		\bcbar[label=4.4 KitKat]{10}
		\bcbar[label=5.0 Lollipop]{4.44}
		\bcbar[label=7.1 Nougat]{3.58}
		\bcbar[label=8.0 Oreo]{2.75}
		\bcbar[label=4.2 Jelly Bean]{1.88}
		\bcbar[label=4.1 Jelly Bean]{0.61}
		\bcbar[label=8.1 Oreo]{0.4}
		\bcbar[label=4.0 Ice Cream Sandwich]{0.32}
		\bcbar[label=4.3 Jelly Bean]{0.27}
		\bcbar[label=2.3 Gingerbread]{0.22}
		\bcbar[label=2.2 Froyo]{0.02}
		\bcbar[label=Other]{0.01}
	\end{bchart}
\end{figure}

Si bien, corresponde a un segmento importante el uso de \emph{Android KitKat} (10\%) es desde la versión de la \emph{API} de \emph{Android} 21, empleada por google en el desarrollo de \emph{Android Lollipop}, que se integra nativamente \emph{Material Design} que es ampliamente recomendado para el diseño de aplicaciones.  Debido a este motivo, se considerá el desarrollo enfocado en sistemas \emph{Android} 5.0 o mayores.

Por otro lado, las versiones de \emph{iOS} y el porcentaje de su participación en el mercado se muestran en el gráfico \ref{chart:iOS Version}.

\begin{figure}[H]
	\centering
	\caption[iOS Version, Market Share in Chile Sept 2017 - Sept 2018.]{iOS Version, Market Share in Chile Sept 2017 - Sept 2018 \\ http://gs.statcounter.com/ios-version-market-share/mobile-tablet/chile/\#monthly-201709-201809-bar.}
	\label{chart:iOS Version}
	\begin{bchart}[step=10, max=100, width=.8\textwidth, unit=\%]
		\bcbar[label=iOS 11.2]{22.93}
		\bcbar[label=iOS 10.3]{19.55}
		\bcbar[label=iOS 11.4]{15.31}
		\bcbar[label=iOS 11.3]{10.19}
		\bcbar[label=iOS 11.0]{9.42}
		\bcbar[label=iOS 11.1]{8.11}
		\bcbar[label=iOS 9.3]{5.39}
		\bcbar[label=iOS 10.2]{3.31}
		\bcbar[label=iOS 7.1]{1.55}
		\bcbar[label=iOS 10.1]{0.83}
		\bcbar[label=iOS 10.0]{0.72}
		\bcbar[label=iOS 12.0]{0.48}
		\bcbar[label=iOS 9.2]{0.34}
		\bcbar[label=Other]{1.87}
	\end{bchart}
\end{figure}

En base al porcentaje de uso de las distintas versiones de \emph{iOS} del gráfico \ref{chart:iOS Version}, se considera un desarrollo para la versión 10.2 o superiores.

\subsubsection{Objetivos cuantificables}

\begin{itemize}
	\item
	Correcta ejecución en una versión minima de \emph{Android}.
	\item
	Correcta ejecución en una versión minima de \emph{iOS}.
\end{itemize}

\subsubsection{Criterios de Validación}

\begin{table}[H]
    \caption[Validación para los indicadores de adapabilidad.] {Validación para los indicadores de adapabilidad.}
    \label{tbl:Criterios de Validación Adapabilidad}
    \begin{tabular}{|p{.6\textwidth}|p{.34\textwidth}|}
        \hline
        \textbf{Criterio} &  \textbf{Medición}\\
    	\hline
    	\hline
    	versión \emph{Android}. & 5.0 o superior. \\ \hline
		versión \emph{iOS}.  & 10.2 o superiores. \\ 
        \hline
    \end{tabular}
\end{table}

\subsection{Temporización}

Se considera la velocidad de conexión del sistema a bordo de 9,6 kbps correspondiente a tecnología \emph{GSM} y una de hasta 2 mbps para una red \emph{3G} para la aplicación móvil, como la velocidad necesaria para un óptimo funcionamiento del sistema.

Debido a que \emph{Gerprin} es una aplicación basada en servicios, es necesario que estos tengan una baja taza de espera al consultar al módulo que contiene la \emph{API}.

\subsubsection{Objetivos cuantificables}

\begin{itemize}
	\item
	Rápida carga de plantillas.
	\item
	Rápida Respuesta de los servicios.
\end{itemize}

\subsubsection{Criterios de Validación}

\begin{table}[H]
    \caption[Validación para los indicadores de temporización.] {Validación para los indicadores de temporización.}
    \label{tbl:Criterios de Validación temporización}
    \begin{tabular}{|p{.6\textwidth}|p{.34\textwidth}|}
        \hline
        \textbf{Criterio} &  \textbf{Medición}\\
    	\hline
    	\hline
    	Tiempo de respuesta.  & inferior a 3 segundos. \\ \hline
		Compresión \emph{gzip}   & Si \\ 
        \hline
    \end{tabular}
\end{table}
Los criterios de validación se basan en la experiencia de aplicaciones similares.

\subsection{Soportar alta carga}

Debido al alto tráfico del servidor central de la aplicación y que esta basa su arquitectura en servicios provistos por una \emph{API}, es de suma importancia que cuente con el soporte necesario para responder a una gran cantidad de solicitudes en un corto periodo de tiempo.

Mediante herramientas como la consola de \emph{Google Play} o \emph{ Google Analitycs} se pueden obtener estadísticas del uso del sistema, tales como tiempo medio de navegación, número de terminales con la aplicación instalada, entre otros factores.

Esta información se emplea para calcular el promedio de usuarios conectados y se estima, en base a aplicaciones similares, que cada uno de ellos realizará una consulta al servidor cada 2 segundos.

Es importante considerar que el número de usuarios varia dependiendo de la popularidad de \emph{Gerprin}. Debe ser obtenido inicialmente mediante un estudio de mercado y posteriormente corregida con datos empíricos.


\subsubsection{Objetivos cuantificables}

\begin{itemize}
	\item
	Cantidad de solicitudes por segundo.
\end{itemize}

\subsubsection{Criterios de Validación}

\begin{table}[H]
    \caption[Validación para los indicadores de soporte de alta carga.] {Validación para los indicadores de soporte de alta carga.}
    \label{tbl:Criterios de Validación soporte de alta carga}
    \begin{tabular}{|p{.6\textwidth}|p{.34\textwidth}|}
        \hline
        \textbf{Criterio} &  \textbf{Medición}\\
    	\hline
    	\hline
    	Cantidad de solicitudes por segundo.  & 1 por segundo cada 2 usuarios.  \\ \hline
    \end{tabular}
\end{table}
Los criterios de validación se basan en la experiencia de aplicaciones similares.

\subsection{Disponibilidad}

Debido a la importancia de la aplicación es completamente necesario que esta se encuentre disponible el mayor tiempo posible.

Los usuarios deben tener plena confianza en que la plataforma se encuentra disponible y enviará las notificaciones \emph{push} si el sistema detecta un intento de robo, ya sea un intento de encender el vehículo por un usuario no autorizado o si este se desplazó sin autorización.

Además, debe estar disponible en casos de ser necesario desactivar el cortacorriente de forma remota sin una huella digital cuando ocurra una emergencia.

Es deseable, además, que la aplicación se encuentre offline por periodos no mayores a 4 horas en caso de presentar algún problema.

Basado en criterios de experiencias en el desarrollo de aplicaciones similares, es necesario que esta no presente un tiempo offline mayor a la suma de 12 horas mensuales y no puede ser en intervalos mayores a 4 horas desde reportado el problema.

\subsubsection{Clúster de servidores}

La aplicación debe encontrarse en un \emph{Cluster} de servidores, es decir, debe encontrarse en varios computadores respondiendo las solicitudes como uno solo. Esto, con el objeto de dividir la carga de procesamiento de los servidores, evitar la falta de disponibilidad por fallas de hardware y permitir escalar la cantidad de carga que puede aguantar el servicio de forma rápida y fácil. Por esta razón el sistema deberá contar con un mínimo de dos instancias de máquinas con la posibilidad de aumentar dicha cantidad según la demanda.

\subsubsection{Objetivos cuantificables}

\begin{itemize}
	\item
	Tiempo online total.
	\item
	Máximo tiempo offline seguido.
\end{itemize}

\subsubsection{Criterios de Validación}

\begin{table}[H]
    \caption[Validación para los indicadores de disponibilidad.] {Validación para los indicadores de disponibilidad.}
    \label{tbl:Criterios de Validación disponibilidad}
    \begin{tabular}{|p{.5\textwidth}|p{.44\textwidth}|}
        \hline
        \textbf{Criterio} &  \textbf{Medición}\\
    	\hline
    	\hline
    	Tiempo online total.  & 98\% del tiempo se debe encontrar online.  \\ \hline
    	Máximo tiempo offline seguido.  & Intervalos no mayores a 4 horas.  \\ \hline
    \end{tabular}
\end{table}
Los criterios de validación se basan en la experiencia de aplicaciones similares.

\subsection{Seguridad}

Debido a la naturaleza del sistema y la información sensible que esta puede manejar, es necesario implementar roles para los usuarios con el propósito de establecer un acceso consistente y seguro a la información contenida en la plataforma. 

Para implementar lo anterior es necesario establecer cuentas de usuarios protegidas por contraseñas que deben ser almacenadas de forma encriptada y un token o \emph{API Key} que permita realizar consultas seguras a la API del sistema.

Es necesario, además, contemplar ataques de tipo \emph{SQL injection}\footnote{http://www.revistasbolivianas.org.bo/pdf/rits/n8/n8a17.pdf}  y de fuerza bruta.


\subsubsection{Limitar puntos de acceso al servidor}

Es necesario establecer la arquitectura del servidor a usar. Se empleará un servidor \emph{AWS EC2} y los puertos abiertos son: 
\begin{itemize}
	\item
	Conexión SSH: puerto 22.
	\item
	Puerto API: 80.
\end{itemize}

\subsubsection{Objetivos cuantificables}

\begin{itemize}
	\item
	Seguridad de sesión.
	\item
	Limitar puntos de acceso al servidor.
\end{itemize}

\subsubsection{Criterios de Validación}

\begin{table}[H]
    \caption[Validación para los indicadores de Seguridad.] {Validación para los indicadores de Seguridad.}
    \label{tbl:Criterios de Validación Seguridad}
    \begin{tabular}{|p{.6\textwidth}|p{.34\textwidth}|}
        \hline
        \textbf{Criterio} &  \textbf{Medición}\\
    	\hline
    	\hline
    	Máximo de intentos fallidos.	& 10.  \\ \hline
    	Ataque fuerza bruta. & Se añade Captcha al formulario al superar máximo de intentos fallidos.  \\ \hline
		Ataque inyección SQL. & No permitir instrucciones de base de datos en ningún formulario.  \\ \hline
		Requisitos al crear contraseña. & Mínimo 8 caracteres con al menos un digito.  \\ \hline
		Encriptar contraseña. & Si, empleando método blowfish.  \\ \hline
    \end{tabular}
\end{table}
Los criterios de validación se basan en la experiencia de aplicaciones similares.

\subsection{Escalabilidad}

Es necesario que el sistema permita añadir nuevas funcionalidades, modificar las ya implementadas, o bien, aumentar la capacidad de trabajo sin afectar el rendimiento de este. 

Para lograr lo anterior es necesario plantear una serie de puntos que permitan agregar o modificar funciones del sistema sin afecten a otras. 

\subsubsection{Objetivos cuantificables}

\begin{itemize}
	\item
	Encapsulamiento de los módulos.
	\item
	Facilitar escalabilidad de la arquitectura del servidor.
\end{itemize}

\subsubsection{Criterios de Validación}

\paragraph{Encapsulamiento de los módulos\\}

Lo primero que debe ser tomado en cuenta es la creación del sistema de forma modular. Esto se logra encapsulando funciones, para que trabajen de manera privada y que solamente puedan ser llamadas por otras funciones sin modificar la forma en que realizan sus procesos. De este modo, se pueden modificar de forma independiente. 

Para este desarrollo, se considera que la aplicación se divide en tres módulos que trabajan juntos, pero su funcionamiento se encuentra encapsulado. El primer módulo es una aplicación móvil desarrollada de forma nativa para \emph{Android} y \emph{iOS}, el segundo módulo es una \emph{API} instalada en un servidor \emph{AWS EC2}. Y finalmente un módulo Arduino a bordo del vehículo. 

\paragraph{Encapsulamiento de los módulos\\}

Otro punto importante que tomar en cuenta es la escalabilidad vertical y horizontal de la arquitectura del servidor. La escalabilidad vertical permite aumentar la capacidad física de almacenamiento y procesamiento, aumentando las características del hardware donde se encuentra el sistema. Mientras que la horizontal permite aumentar el número de nodos que conforman el \emph{Cluster} del servidor. Para el caso de \emph{Gerprin}, \emph{AWS EC2} permite la rápida implementación de ambos métodos.

\subsection{Mantenibilidad}

\emph{ISO/IEC 25010}\footnote{I. S. Organization, “ISO/IEC 25010,” in Systems and software engineering - Systems and software Quality Requirements and Evaluation (SQuaRE) - System and Software Quality Models, ed, 2011} define la mantenibilidad como el grado de efectividad o eficiencia con la que un producto o sistema puede ser modificado. Para logar reducir el trabajo de mantenibilidad del \emph{software} existen una variedad de atributos que pueden ser integrados para facilitar la labor. 

\subsubsection{Objetivos cuantificables}

\begin{itemize}
	\item
	Comentarios en el código fuente.
	\item
	Estandarización del nombre.

\end{itemize}

\subsubsection{Criterios de Validación}

\paragraph{Comentarios en el código fuente\\}

Es necesaria la existencia de comentarios dentro del código fuente del programa. Se busca que mínimamente se encuentren comentarios de cada función y método indicando valores de entrada, salida y una descripción de las acciones que realiza. Además, se esperan comentarios explicativos de cada clase indicando el significado de sus atributos. 

\paragraph{Estandarización del nombre\\}

Otra buena práctica para facilitar el mantenimiento es la estandarización de nombres siguiendo la convención que existe para este propósito para \emph{PHP}\footnote{http://pear.php.net/manual/en/standards.naming.php}.
