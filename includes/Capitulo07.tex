\chapter{Conclusiones }
Para el aseguramiento de la calidad de \textit{Gerprin} se debe evaluar no solo el resultado de los artefactos, además, deben ser puesto bajo análisis los procesos conducentes a la elaboración de dichos productos. 
En el sistema presentado en este documento, se detalla el uso de un desarrollo incremental basado en \textit{Extreme Programming}. Dicha estructura de desarrollo provee dos herramientas potentes que facilitan el aseguramiento de calidad.\begin{itemize}
\item
La primera herramienta a utilizar es \textit{Test-Driven Development (TDD)}, que permite el desarrollo del sistema en base al diseño de pruebas, las que deben  ser detalladas antes de comenzar el proceso de codificación. Esta herramienta permite asegurar la calidad del \textit{software} desde etapas tempranas del desarrollo y que errores iniciales escalen a etapas finales del sistema. 
TDD cambia la visión del desarrollo al iniciar desde las pruebas y construir el sistema desde ellas. Dicho cambio de estructura es fundamental para comprender de forma sistémica el desarrollo del proyecto y establecer como objetivo el aseguramiento de calidad, no solo desde un punto de vista técnico, sino instaurando en el equipo de desarrollo como una característica fundamental del sistema 
\item
En segundo lugar, se emplea diseño emergente que permite tomar decisiones respecto al diseño estructural del sistema cuando es necesario y no antes, permitiendo una mayor flexibilidad al momento del desarrollo.
Un diseño inicial detallado no presenta una coherencia estructural con la metodología utilizada y presenta un enfoque más próximo a un modelo en cascada. Debido a la rigidez del diseño detallado se pierden ventajas de las iteraciones del desarrollo que permiten una mejora continua incluso dentro del mismo ciclo de construcción del sistema; Se ven menoscabadas la inclusión de tecnologías o estrategias emergentes de iteraciones anteriores y permite un limitado margen de acción ante anomalías detectadas de forma tardía en el diseño. 
\end{itemize}
En relación con lo mencionado anteriormente, es el área de SQA quienes se deben encargar de la elaboración de estándares, guías y evaluación de herramientas que aseguren que las acciones realizadas para cada una de las etapas del desarrollo de software sean las correctas, con el objetivo de asegurar la calidad de los distintos artefactos elaborados a lo largo de todo el proceso definido por \textit{Extreme Programming}. Con el mismo objetivo anterior, el departamento de aseguramiento de calidad, debe realizar revisiones a los productos propiamente antes de ser liberados por completo, asegurando que estos sean coherentes entre si y cumplan con los objetivos planteados inicialmente. 
El equipo de \textit{SQA} cuenta con tres métodos para detectar problemas en el desarrollo y facilitar las actividades antes mencionadas.
\begin{itemize}
\item
Las revisiones son una metodología que busca detectar problemas de forma temprana en el desarrollo mediante la inspección de procesos y productos, convirtiéndose en una actividad fundamental para el aseguramiento de calidad en la detección de cualquier tipo de anomalías en productos o procesos. 
\item
SQA, cuenta también, con las auditorias que se encargan de comparar el estado real del proyecto con el reportado, permitiendo una retroalimentación en el desarrollo. Las auditorias cuentan como una actividad fundamental para la detección de incidentes directamente relacionadas por el equipo de trabajo, motivo por el cual pueden ser implementadas por un grupo externo a este de carácter imparcial. 
La naturaleza de las auditorias guardan una estrecha relación entre las características claves de un equipo de \textit{Extreme Programming}. La inspección de las diferencias entre el estado reportado del proyecto y el real se relaciona directamente con la característica de valor que es necesaria en un equipo bajo dicha lógica.
\item
\textit{Check list} corresponde a una herramienta fundamental en cada una de las etapas del desarrollo de software asegurando el cumplimiento de las actividades o acciones que deben ser llevadas a cabo por los distintos actores del proyecto, además, permite rastrear el origen de incidentes aun cuando su detección es tardía o no se relaciona directamente a su origen
\end{itemize}
Finalmente, respecto a las pruebas que deben ser ejecutadas sobre el sistema, se debe mencionar que como integrantes de un equipo que trabaja en un proyecto de TI es fundamental la implementación de herramientas que permitan la automatización de labores, la introducción de pruebas automáticas permite disminuir errores en el proceso de desarrollo y realizarlas en un tiempo significativamente menor. 
