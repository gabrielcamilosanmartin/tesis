%!TEX root = memoria.tex

%---------------------------------------------------------------------------
%%% CONFIGURACIÓN
%---------------------------------------------------------------------------
%
%%% CODIFICACIÓN DE CARACTERES
% Este documento está escrito usando caracteres Unicode (UTF8)
% Por lo que la siguiente línea es necesaria para reconocer los acentos
% y otros caracteres en español.
% Si ve caracteres extraños en el PDF (en Windows o MAC) pruebe 
% con alguna de estas líneas:
\usepackage[utf8x]{inputenc}    % *nix / Linux / MacOSX
%\usepackage[latin1]{inputenc}  % Windows (MacOSX)


\newcommand{\TheTitle}           {PLAN DE CALIDAD PARA GERPRIN}
\newcommand{\TheAuthor}          {GABRIEL ARÍSTIDES CAMILO SAN MARTÍN}
\newcommand{\TheGrade}           {INGENIERO DE EJECUCIÓN EN INFORMÁTICA}
\newcommand{\TheCity}            {SANTIAGO}
\newcommand{\TheDate}            {ABRIL 2019}
\newcommand{\TheAdvisor}         {MARCELLO VISCONTI ZAMORA}
\newcommand{\TheCoAdvisor}       {LIOUBOV DOMBROVSKAIA}

% Marca de agua, puede ser deshabilitada para impresión rápida
\insertWatermark{figures/logousm_watermark.jpg}
%---------------------------------------------------------------------------


%---------------------------------------------------------------------------
%%% No editar (¡Ver licencia!) (MIT License, 2016)
%---------------------------------------------------------------------------
\hypersetup{  
    pdfinfo={  
        Subject={Memoria Departamento de Industria, UTFSM},
        Keywords={Memoria} {Departamento de Industrias} {UTFSM},
        Producer={JCR LaTeX Templates, http://www.rubin-de-celis.com/},
        Licence={http://www.rubin-de-celis.com/LICENSE},
        pdfpagemode=FullScreen,
        pdfmenubar=false,
        pdftoolbar=false
    }  
}
\hypersetup{  
    pdfinfo={  
        Title={\TheTitle},
        Author={\TheAuthor}
    }  
}
%---------------------------------------------------------------------------