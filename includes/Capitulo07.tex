%!TEX root = ../memoria.tex

\chapter{Conclusiones }

En la presente documento, se aborda el aseguramiento de la calidad de \emph{Gerprin}, sin embargo, bajo un primer análisis se puede apreciar que es imposible asegurar objetivamente que un sistema posee una característica tan abstracta como el concepto de “calidad”. Para poder asegurar la calidad de un sistema es fundamentar definir los indicadores que al ser cumplidos permiten definir \emph{Gerprin} como un software de calidad.

La elección de los requisitos no funcionales y de los atributos de calidad es fundamental para el aseguramiento de esta, pues son estos, que, mediante su cumplimiento, definirán la calidad. En la selección de los indicadores que marcan la calidad de software se destaca la clasificación del modelo ISO 9126 que clasifica en 6 grandes grupos las características que debe tener un software para considerarse que es de “calidad”. En el presente documento se consideraron dichos elementos, sin embargo, en la sección de requisitos no funcionales, se destacó un grupo reducido de indicadores sobre el resto debido a la experiencia en el desarrollo de software similares, donde primaban los atributos de calidad seleccionados por sobre otros.

Más allá de definir el concepto de calidad, al cual se busca llegar, es necesario establecer y definir un proceso por el cual se logre alcanzar los estándares impuestos. Por este motivo la segunda etapa de un plan de aseguramiento de calidad es definir las metodologías que serán empleadas para el desarrollo del software. 

Como se hizo énfasis, \emph{Gerprin} es un sistema que ya se encuentra desarrollado. En esta ocasión se apunta a una nueva versión, sin desconocer la actual. Por este motivo se pueden considerar metodologías menos ortodoxas, como el concepto de diseño emergente, que fue descrito en la sección correspondiente. Esta técnica corresponde a una característica de la metodología de \emph{XP} que resalta por su gran flexibilidad en el desarrollo, sobre esto se aplica \emph{TDD} que permite la temprana detección de errores, que sumado a la flexibilidad de \emph{XP} permiten una pronta solución y evolución del software.

Asimismo, se implementó un proceso de \emph{SQA} basado en lo descrito en \citet{web00} que clasifica a las organizaciones que utilizan \emph{SQA} de alguna u otra forma en el desarrollo de software según cuatro niveles. Los niveles de madurez se detallan en base a los procesos y subprocesos que se implementan. Para el sistema aquí presentado se internalizaron los estándares, procesos y actividades recomendadas hasta el segundo nivel, que representa el núcleo del aseguramiento de calidad. En el desarrollo de \emph{Gerprin} se busca simplificar los procesos dejando de lado largas documentaciones o procesos que aumentan los tiempos de desarrollo. 

El primer nivel de madurez se integró reconociendo la importancia del proceso de \emph{SQA} y su integración al desarrollo del software. Mediante el presente documento se definen las acciones que procederán a integrar el proceso, es decir, de adoptan las revisiones y auditorias, que mediante checklist mantienen un control sobre la adherencia de los productos y procesos a los estándares y procedimientos establecidos.

El segundo nivel es lograr el compromiso de la organización, mediante políticas organizacionales, con el área de \emph{SQA} para incorporar un plan de \emph{SQA} en cada proyecto y establecer los responsables de las actividades de aseguramiento de calidad. Para \emph{Gerprin}, se establece un área de \emph{SQA} y los responsables de las actividades relacionadas al aseguramiento de la calidad. 

Una vez realizado el proceso de desarrollo empleando las metodología, herramientas y esquema organizacional definidas en el plan de calidad, es necesario comprobar que los indicadores definidos en las etapas iniciales del proyecto se cumplen. Mediante la etapa de pruebas se asegura que el sistema cumpla con los estándares impuestos que aseguran la calidad del producto final, En la sección correspondiente se detallan las pruebas que se realizan para asegurar el cumplimiento de todos los requerimientos. 

No establecer estándares que deben ser cumplidos entrega un producto que puede o no tener la calidad necesaria para un uso productivo; lo mismo ocurre al no establecer las metodologías y herramientas, creando un camino que es incapaz de asegurar el cumplimiento de los estándares definidos inicialmente. Asegurar la calidad de un producto es una labor integral que debe considerar el proceso interno y no solo el resultado final, comprender que es lo necesario para obtener un software de calidad y los pasos para conseguirlo son las etapas tempranas en un desarrollo exitoso. 

La aparición de nuevas metodologías que se adaptan a los procesos de \emph{SQA} son esenciales en el desarrollo de cualquier producto. En el caso puntual del desarrollo de \emph{Gerprin} la implementación de \emph{TDD} sobre \emph{XP} permite una detección temprana de los errores de codificación y tener espacios para gestionar una solución. 

Lejos estamos de la época conocida como “crisis del software”, donde el desarrollo de sistemas era un proceso relativamente nuevo, sin metodologías y estándares que permitan asegurar la calidad del producto final. Herramientas como checklist, automatización de pruebas, máquinas virtuales, entre otras, no solo permiten asegurar lo que años antes no se podía, además lo logran con un menor esfuerzo humano, sin embargo, para lograr esto es necesario la planificación e integración de dichas técnicas y herramientas en el proceso de ingeniería de software. 
