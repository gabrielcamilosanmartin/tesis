%!TEX root = ../memoria.tex

\chapter{Introducción}
Considerando el prototipo de \textit{Gerprin} desarrollado para la XXIII Feria de Software de la Universidad Técnica Federico Santa María en año 2015, se busca crear un plan de calidad para el desarrollo de una nueva versión de este con propósitos comerciales, y así lograr un producto con estándares de calidad adecuados al mercado y asegurar un correcto funcionamiento de la nueva versión del sistema.

Es menester comenzar contextualizando el problema que se busca solucionar mediante la invención de \textit{Gerprin}, antes de proceder con una descripción preliminar del plan de calidad que se busca realizar. 

\section{Resumen del proyecto}

El Informe Anual entregado por Carabineros de Chile (Chile, 2018) contabilizó un total de 16.678 robos de vehículos motorizados entre los meses de enero y agosto. Esto significa que se sustrajo un vehículo, en nuestro país, cada 21 minutos.

En respuesta a lo anterior, nace \textit{Gerprin}, él busca eliminar la inseguridad en los conductores y permitir que puedan aparcar en cualquier lugar sin miedo a ser víctimas del robo de su vehículo. Para lograr esto, fue propuesto un sistema de seguridad vehicular integral que cuenta con tres aspectos. En primer lugar, un sistema a bordo de corta corriente controlado con un lector biométrico para que ningún extraño pueda operar el vehículo. En segundo lugar, un sistema centralizado de monitoreo de la posición \textit{GPS} del vehículo que genera alertas al cliente cuando se detectan movimientos no autorizados. Y finalmente, se desarrolló una aplicación móvil que le permitirá a nuestros clientes monitorear la posición de su vehículo y controlar el sistema a bordo, permitiendo agregar, modificar o eliminar usuarios de una forma fácil y rápida.

 Se buscó solucionar una de las necesidades más básicas del ser humano: la seguridad. No sólo se creó un gadget antirrobo, sino que gestó un completo sistema de protección para el automóvil, el cual entregará confianza y tranquilidad a los usuarios cuando deban estacionar sus vehículos en lugares que no conocen o incluso en su propio hogar. Con \textit{Gerprin} de monitoreo podrán saber en todo momento cual es la posición de su automóvil, mientras que, con el sistema biométrico, tendrán la certeza de que nadie podrá utilizar el vehículo sin su autorización.

 En último lugar, para comprender la envergadura del sistema es necesario señalar las diferencias con los productos del mercado. Señalar, además, que ya existen sistemas de corta corriente, incluso los hay operados con biometría; Los sistemas de monitoreo \textit{GPS} ya se han masificado y especialmente en el área de transporte comercial. \textit{Gerprin} representa un sistema de seguridad integral que combina todos los aspectos anteriores: la seguridad a bordo (corta corriente con biometría), monitoreo satelital centralizado (\textit{GPS}) y el control de estos mediante una aplicación móvil fácil de utilizar.

\section{Propósito}

Como fue mencionado al comienzo del presente documento, este trabajo busca realizar una segunda iteración en el desarrollo de \textit{Gerprin}, con el fin de generar una versión comercializable del sistema. Utilizando como punto de partida la primera iteración diseñada y desarrollada para la XXIII feria de software de la Universidad Técnica Federico Santa María.

En el presente plan de calidad se tiene como propósito establecer los procesos y acciones para el aseguramiento de calidad, definiendo la estructura para el desarrollo del sistema, las gestiones necesarias para dicho desarrollo, las herramientas y técnicas a emplear y las pruebas necesarias para asegurar la calidad esperada.

\section{Alcance}

El plan de calidad abordará las distintas etapas de ingeniería de software, desde un punto de vista práctico, enfocado a los productos y desarrollo de los pasos de análisis de requerimientos, diseño, desarrollo pruebas y documentación.

Dada la existencia previa de un software funcional desarrollado el año 2015 que cumple con los requisitos funcionales especificados para dicha aplicación, el presente plan de calidad se enfocará en los requisitos no funcionales.